\documentclass[11pt, aspectratio=169, modernfonts]{beamer}
%-----------------------------------------------------------------------
% LaTeX packages
%-----------------------------------------------------------------------
\usepackage[utf8]{inputenc}
\usepackage{amsmath}
\usepackage{amssymb}
\usepackage[ngerman]{babel}
\usepackage{beamerprosper}
\usepackage{color}
\usepackage{stackengine,graphicx}

\usepackage{latexsym}
\usepackage{listings}
\usepackage{times}
\usepackage{url}
\usepackage{xspace}
\usepackage{booktabs}
\usepackage{svg}
\usepackage{caption}
\usepackage{wrapfig}
\usepackage{algorithmicx}
\usepackage{algorithm}
\usepackage{multicol}
\usepackage[framemethod=TikZ]{mdframed}
\usepackage{lstautogobble}
\usepackage{csquotes}

\usepackage[
    backend=biber,
    natbib=true,
    style=numeric,
    sorting=none
]{biblatex}

\addbibresource{sources.bib}

%-----------------------------------------------------------------------
% LaTeX Theme
%-----------------------------------------------------------------------

\usetheme{CambridgeUS}
\definecolor{midnightblue}{rgb}{0,.25,.5}
\definecolor{darkblue}{rgb} {0.125,0.25, 0.625}
\definecolor{darkgreen}{rgb}{0.125,0.625,0.25}
%\definecolor{darkred}{rgb}  {0.625,0.125,0.25}
\definecolor{darkred}{rgb}  {0.0,0.4,0.59}

\mode<presentation>
{
\useinnertheme{rectangles}
\setbeamertemplate{navigation symbols}{}
}

\setbeamercolor{footlinecolorl}{fg=black,bg=lightgray}
\setbeamercolor{footlinecolor}{fg=black,bg=gray}
\setbeamercolor{footlinecolord}{fg=black,bg=darkgray}
\setbeamercolor{block title}{bg=darkred,fg=white}

\setbeamertemplate{itemize item}{\color{darkred}$\blacksquare$}
\setbeamertemplate{itemize subitem}{\color{darkred}$\blacktriangleright$}
\setbeamercolor{item projected}{bg=darkred}
\setbeamertemplate{enumerate items}[default]
\setbeamercolor{enumerate item}{fg=darkred}
\setbeamercolor{enumerate subitem}{fg=darkred}
\setbeamercolor{enumerate subsubitem}{fg=darkred}

\setbeamertemplate{title page}
{
  \vbox{}
  \vfill
  \begin{centering}
    \begin{beamercolorbox}[sep=8pt,center]{title}
      \usebeamerfont{title}\inserttitle
    \end{beamercolorbox}
    \setbeamercolor{title}{bg=white,fg=darkred}
    \begin{beamercolorbox}[sep=8pt,center]{title}
      {\usebeamercolor[fg]{titlegraphic}\inserttitlegraphic\par}
      \ifx\insertsubtitle\@empty%
      \else%
        \vskip0.25em%
        {\usebeamerfont{title}\usebeamercolor[fg]{title}\insertsubtitle\par}%
      \fi%     
    \end{beamercolorbox}%
    \vskip1em\par
    \begin{beamercolorbox}[sep=8pt,center]{author}
      \usebeamerfont{author}\insertauthor
    \end{beamercolorbox}
    %\vskip-1em\par % change here
    \begin{beamercolorbox}[sep=8pt,center]{institute}
      \usebeamerfont{institute}\insertinstitute
    \end{beamercolorbox}
    \begin{beamercolorbox}[sep=8pt,center]{date}
      \usebeamerfont{date}\insertdate
    \end{beamercolorbox}\vskip0.5em
  \end{centering}
  \vfill
}

\setbeamertemplate{footline}{%
\hbox{%
\begin{beamercolorbox}[wd=.40\paperwidth,ht=4.25ex,left,leftskip=3ex]{author in head/foot}%
    \vbox to4.25ex{\vfil\hbox{\usebeamerfont{author in head/foot} \insertframenumber{} / \inserttotalframenumber \hspace{0.5cm}\insertshortauthor}\vfil}%
\end{beamercolorbox}%
\begin{beamercolorbox}[wd=.30\paperwidth,ht=4.25ex,center]{title in head/foot}%
    \vbox to4.25ex{\vfil\hbox{\usebeamerfont{date in head/foot}\insertshorttitle{}}\vfil}%
\end{beamercolorbox}%
\begin{beamercolorbox}[wd=.30\paperwidth,ht=4.25ex,right,rightskip=3ex]{date in head/foot}%
    \vbox to4.25ex{\vfil\hbox{\insertshortdate{}}\vfil}%
\end{beamercolorbox}}%
}

%\setbeamertemplate{footline}{
%  \quad \tiny \insertshortauthor \hfill \insertshorttitle \qquad \hfill \insertshortdate\ \qquad\qquad
%}

\usefonttheme{default}

%-----------------------------------------------------------------------
% Custom Commands
%-----------------------------------------------------------------------

% Adds a slide with the specified text. It does not show any header or footer.
\newcommand{\finalslide}[1]{%
    \part{} % no section title in footer for this slides
    \begin{frame}[noframenumbering, plain]
        \begin{center}
            \vfill
            \Huge #1
            \vfill
        \end{center}
    \end{frame}
}

% ---------------------------------------------------------------------
% Document Title
% ---------------------------------------------------------------------

\title[Smart Home Hacking]{Bite The Bytes - Informatik studieren in Weimar}
\subtitle{\Large{Smart Home Hacking}}
\author[M. Almert]{Marcus Almert}
\institute[Bauhaus-Universität Weimar]{Bauhaus-Universität Weimar\\[0.5cm] Fakultät Medien}

\date{4. November 2022}
\raggedright
\AtBeginSection{\frame{\sectionpage}}

% ---------------------------------------------------------------------
\begin{document}
% ---------------------------------------------------------------------

    \maketitle

%    \begin{frame}{Table of Contents}
%        \tableofcontents
%    \end{frame}

% ---------------------------------------------------------------------
% Document Content
% ---------------------------------------------------------------------
    \begin{frame}{Vorwort}
        \begin{itemize}
            \item \enquote{Hacken} von dritten ist \textbf{illegal}!
            \item Experimentelle Angriffe ausschließlich auf eigene Geräte und Accounts
            \vspace{0.4cm}
            \item Wieso beschäftigen wir uns mit \enquote{Hacken}?
            \item[$\rightarrow$] Wissenschaftliche Untersuchungen weisen auf Missstände hin und tragen zur Verbesserung bei
        \end{itemize}
    \end{frame}


    \section{Was ist Smart Home?}\label{sec:was-ist-smart-home?}

    \begin{frame}{Was ist Smart Home?}
        \begin{itemize}
            \item Geräte, die miteinander kommunizieren, agieren und zentral gesteuert sind\\[.4cm]
            \item Kommunikation erfolgt meist über Bluetooth oder WIFI\\[.4cm]
            \item Steuerung durch Apps und Sprachassistenten\\[.4cm]
        \end{itemize}
    \end{frame}

    \begin{frame}{Vorteile von Smart Home}
        \begin{itemize}
            \item Erhöhung der Lebens- und Wohnqualität (smarte Glühbirne)\\[.4cm]
            \item Steigerung der Energieeffizienz (smarte Thermostate)\\[.4cm]
            \item Verbesserung der Sicherheit (smarte Überwachungskameras)\\[.4cm]
        \end{itemize}
    \end{frame}

    \begin{frame}{Smarte Türklingeln}
        \begin{columns}
            \begin{column}{0.6\textwidth}
                \begin{itemize}
                    \item Bestandteile:
                    \begin{itemize}
                        \item Kamera
                        \item Bewegungssensor
                        \item Mikrofon und Lautsprecher
                        \item Klingeltaste
                    \end{itemize}
                    \vspace{0.2cm}
                    \item Funktionen:
                    \begin{itemize}
                        \item Fernzugriff via App (Account notwendig)
                        \item Live Video und Ton Stream
                        \item Benachrichtigungen bei Bewegungen
                        \item Anruf auf Smartphone bei Klingeln
                        \item Automatisches Aufnehmen von Fotos und Videos
                    \end{itemize}
                \end{itemize}
            \end{column}

            \begin{column}{0.4\textwidth}
                \begin{figure}
                    \centering
                    \includegraphics[width=0.5\linewidth]{images/doorbell}\label{fig:doorbell}
                \end{figure}
            \end{column}
        \end{columns}
        \hfill [1]
    \end{frame}


    \section{Smart Home Hacking}\label{sec:smart-home-hacking}

    \begin{frame}{Ziele und Vorgehensweise}
        \begin{itemize}
            \item Ziele:
            \begin{itemize}
                \item Unerlaubter Zugriff auf Ressourcen (Daten und Funktionen)
                \item Denial of Service
                \item Einschleusen von Schadsoftware (z.B.~Viren, Trojaner, Ransomware)
                \item Infiltration von weiteren Geräten
            \end{itemize}
            \item Vorgehensweise:
            \begin{itemize}
                \item System und Gerät Verstehen
                \item Angriffspunkte und Schwachstellen Identifizieren
                \item Angriffe Ausführen
                \item Auswertung
            \end{itemize}
        \end{itemize}
    \end{frame}

    \begin{frame}{Vorgehensweise am Beispiel von smarten Türklingeln}
        \begin{figure}
            \includegraphics[height=0.8\textheight]{images/function.drawio}\label{fig:function}
        \end{figure}
    \end{frame}

    \begin{frame}{Person-in-the-middle Attacke}
        \begin{figure}
            \includegraphics[height=0.8\textheight]{images/pitm.drawio}\label{fig:pitm}
        \end{figure}
    \end{frame}

    \begin{frame}[fragile]{Ergebnis der Person-in-the-middle Attacke}
        \begin{columns}
            \begin{column}{0.5\textwidth}
                Kommunikation verletzt Prinzipien der Sicherheit:
                \begin{itemize}
                    \item Vertraulichkeit
                    \item Integrität
                    \item Authentizität
                \end{itemize}
            \end{column}
            \begin{column}{0.5\textwidth}
            \end{column}
        \end{columns}
    \end{frame}

    \begin{frame}[fragile]{Ergebnis der Person-in-the-middle Attacke}
        \begin{columns}
            \begin{column}{0.5\textwidth}
                Kommunikation verletzt Prinzipien der Sicherheit:
                \begin{itemize}
                    \item Vertraulichkeit
                    \item Integrität
                    \item Authentizität
                \end{itemize}
            \end{column}
            \break
            \begin{column}{0.5\textwidth}
                Beispiel für abgehörte Daten:
                \begin{lstlisting}[basicstyle=\ttfamily, label={lst:sign-in}]
POST /login HTTP/2.0
    Host:       api.gdxp.com
    username:   <zensiert>
    password:   <zensiert>
                \end{lstlisting}
            \end{column}

        \end{columns}
    \end{frame}

    \begin{frame}{Auswertung des Angriffs I}
        \textbf{Teil 1: Auf was haben wir unerlaubten Zugriff?}\\[0.2cm]
        \begin{itemize}
            \item Steuerungsaccount
            \item[$\rightarrow$] Fotos und Videos im Cloudspeicher
            \item[$\rightarrow$] Live Video und Audio Stream
            \item[$\rightarrow$] Geräte und Account Einstellungen
        \end{itemize}
    \end{frame}


    \begin{frame}{Auswertung des Angriffs II}
        \textbf{Teil 2: Was können wir damit machen?}\\[0.2cm]
        \begin{itemize}
            \item Identifizierung des Ortes (durch Kamera, WIFI, IP-Adresse)
            \item Bewegungsprofil Erstellen: Wann ist wer wo?
            \item Accounteinstellungen: Passwort ändern)
            \item Geräteeinstellungen (Kamera und Mikrofon ausschalten)
            \item Cloudspeicher: Diebstahl und Löschen von Daten,
            \item Zugriff zu Accounts mit gleichen oder ähnlichen Zugangsdaten
        \end{itemize}
    \end{frame}

    \begin{frame}{Fazit}
        \begin{itemize}
            \item Smart Home Geräte können viele Sicherheitsrisiken mit sich bringen
            \vspace{0.4cm}
            \item Es herrscht eine riesige Intransparenz was mit online gesammelten Daten passiert
            \vspace{0.4cm}
            \item[$\rightarrow$] Sicherheit der Geräte sollte nicht überschätzt werden
        \end{itemize}
    \end{frame}

    \finalslide{Vielen Dank für Eure Aufmerksamkeit!\\[1cm] Gibt es Fragen?}


% ---------------------------------------------------------------------
% Bibliography
% ---------------------------------------------------------------------

%\begin{frame}[noframenumbering,allowframebreaks]{Sources}
%    \printbibliography[title = {Sources}, heading = none]
%\end{frame}

\end{document}
