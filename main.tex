\documentclass[11pt, aspectratio=169, modernfonts]{beamer}
%-----------------------------------------------------------------------
% LaTeX packages
%-----------------------------------------------------------------------
\usepackage[utf8]{inputenc}
\usepackage{amsmath}
\usepackage{amssymb}
\usepackage[ngerman]{babel}
\usepackage{beamerprosper}
\usepackage{color}
\usepackage{graphicx}
\usepackage{latexsym}
\usepackage{listings}
\usepackage{times}
\usepackage{url}
\usepackage{xspace}
\usepackage{booktabs}
\usepackage{svg}
\usepackage{caption}
\usepackage{wrapfig}
\usepackage{algorithmicx}
\usepackage{algorithm}
\usepackage{multicol}
\usepackage[framemethod=TikZ]{mdframed}
\usepackage{lstautogobble}
\usepackage{csquotes}

\usepackage[
backend=biber,
natbib=true,
style=numeric,
sorting=none
]{biblatex}

\addbibresource{sources.bib}

%-----------------------------------------------------------------------
% LaTeX Theme
%-----------------------------------------------------------------------

\input{theme.tex}
\usefonttheme{default} 

%-----------------------------------------------------------------------
% Custom Commands
%-----------------------------------------------------------------------

% Adds a slide with the specified text. It does not show any header or footer.
\newcommand{\finalslide}[1]{%
  \part{} % no section title in footer for this slides
  \begin{frame}[noframenumbering, plain]
    \begin{center}
      \vfill
      \Huge #1
      \vfill
    \end{center}
  \end{frame}
}

% ---------------------------------------------------------------------
% Document Title
% ---------------------------------------------------------------------

\title[Bite the Bytes]{Bite The Bytes - Informatik studieren in Weimar}
\subtitle{\Large{Smart Home Hacking}}
\author[M. Almert]{Marcus Almert}
\institute[Bauhaus-Universität Weimar]{Bauhaus-Universität Weimar\\[0.5cm] Fakultät Medien}

\date{4. November 2022}
\raggedright
\AtBeginSection{\frame{\sectionpage}}

% ---------------------------------------------------------------------
\begin{document}
% ---------------------------------------------------------------------

\maketitle

\begin{frame}{Table of Contents}
    \tableofcontents
\end{frame}

% ---------------------------------------------------------------------
% Document Content
% ---------------------------------------------------------------------

\section{Was ist smart Home?}\label{sec:was-ist-smart-home?}

\begin{frame}{Was ist smart Home}
  \begin{itemize}
    \item Geräte, die miteinander kommunizieren, agieren und zentral gesteuert sind
    \item Kommunikation erfolgt meist über Bluetooth oder WIFI
    \item Steuerung durch Apps und Sprachassistenten
  \end{itemize}
\end{frame}

\begin{frame}{Vorteile von smart Home}
  \begin{itemize}
    \item Erhöhung der Lebens- und Wohnqualität durch intelligente Automatisierungen
    \item Steigerung der Energieeffizienz
    \item
  \end{itemize}
\end{frame}

\begin{frame}{Smart Home Beispiele}

\end{frame}

\begin{frame}{Smarte Türklingeln}
  \begin{columns}
    \begin{column}{0.6\textwidth}
      \begin{itemize}
        \item Bestandteile:
        \begin{itemize}
          \item Kamera
          \item Bewegungssensor
          \item Mikrofon und Lautsprecher
          \item Klingeltaste
        \end{itemize}
        \vspace{0.2cm}
        \item Funktionen:
        \begin{itemize}
          \item Live Video und Ton über App
          \item Benachrichtigungen bei Bewegungen
          \item Anruf auf Smartphone bei Klingeln
          \item Automatisches Aufnehmen von Fotos und Videos
        \end{itemize}
      \end{itemize}
    \end{column}

    \begin{column}{0.4\textwidth}
      \begin{figure}
        \centering
        \includegraphics[width=0.5\linewidth]{images/doorbell}\label{fig:doorbell}
      \end{figure}
    \end{column}
  \end{columns}
  \hfill [1]
\end{frame}
  
\section{Smart Home Hacking}

  \begin{frame}{Was bedeutet Hacking?}
    \begin{itemize}
      \item Was sind die Ziele?
      \begin{itemize}
        \item Unerlaubter Zugriff auf Ressourcen (Daten und Funktionen)
        \item Denial of Service
        \item Einschleusen von Schadsoftware (z.B.~Viren, Trojaner, Ransomware)
        \item Infiltration von weiteren Geräten
      \end{itemize}
      \item Was sind die ersten Schritte?
      \begin{itemize}
        \item System und Gerät Verstehen
        \item Angriffspunkte und Schwachstellen Identifizieren
        \item
      \end{itemize}
    \end{itemize}
  \end{frame}


% ---------------------------------------------------------------------
% Bibliography
% ---------------------------------------------------------------------

%\begin{frame}[noframenumbering,allowframebreaks]{Sources}
%    \printbibliography[title = {Sources}, heading = none]
%\end{frame}

\end{document}
